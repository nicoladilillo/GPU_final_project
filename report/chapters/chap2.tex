\chapter{Parallelization}
\label{chap2}

The advantage of GPU implementation derive to the possibility to create all possible combinations of resources in a parallel way and not sequentially. Theoretically this means that in a single instant the entire set is created, practically little amount of time pass between one and other and, if the set is too big, parallelization is split in more blocks.

\section{Combination}

Formula \ref{comb_formula} indicates all to possible combinations that can be extract from a set of n elements choosing group of k elements. In this case the n indicate the occurrence of resources. 

\begin{equation}
    \binom{n}{k} = \frac{n!}{k!(n-k)!}
    \label{comb_formula}
\end{equation}

The complete power set to formulate has an occurrence equal to the sum of all combination from the minimal number of resources needed to the max number, like show in formula \ref{power_set}.

\begin{equation}
    \sum_{i=k_{min}}^{k_{max}} \binom{n}{i} = 
    \sum_{i=k_{min}}^{k_{max}} \frac{n!}{i!(n-i)!}
    \label{power_set}
\end{equation}

\section{Combinadic}
\label{Combinadic}

Combinadic is a useful technique that, giving and index in the range between 0 and $\binom{n}{k}-1$ , return a unique combination set of k element.

From a given combination is possible to find the corresponding number $N$, corresponding to 
$c_k > ... > c_2 > c_1$, according formula \ref{id_Combinadic}.
 
\begin{equation}
    N = \binom{c_k}{k} + ... + \binom{c_2}{2} + \binom{c_1}{1}
    \label{id_Combinadic}
\end{equation}

From number $N$ is more difficult extract the combination. By the definition of the lexicographic ordering, two k-combinations that 
differ in their largest element $c_k$ will be ordered according to the comparison of those largest elements, from which it follows that all
combinations with a fixed value of their largest element are contiguous in the list. Moreover the smallest combination with $c_k$ as
the largest element is $\binom {c_{k}}{k}$, and it has $c_i = i - 1$ for all $i < k$ (for this combination all terms 
in the expression except $\binom {c_{k}}{k}$ are zero). 

Therefore $c_k$ is the largest number such that $\binom {c_{k}}{k} \leq N$. 
If $k > 1$ the remaining elements of the k-combination form the (k-1)-combination corresponding to the number
$N - \binom {c_{k}}{k}$ in the combinatorial number system of degree k - 1, and can therefore be found 
by continuing in the same way for $N - \binom {c_{k}}{k}$ and k - 1 instead of N and k.

\subsection{Example}

Suppose one wants to determine the 5-combination at position 72.
The successive values of $\binom {n}{5}$ for n = 4, 5, 6, ... are 0, 1, 6, 21, 56, 126, 252, ..., of which the largest 
one not exceeding 72 is 56, for n = 8. Therefore c5 = 8, and the remaining elements form the 4-combination at position 
72 - 56 = 16. The successive values of $\binom {n}{4}$ for n = 3, 4, 5, ... are 0, 1, 5, 15, 35, ..., of which the largest
one not exceeding 16 is 15, for n = 6, so c4 = 6. Continuing similarly to search for a 3-combination at position 16 1 15 = 1
one finds c3 = 3, which uses up the final unit.
This establishes $72=\binom{8}{5}+\binom{6}{4}+\binom {3}{3}$, and the remaining values $c_i$ will be the maximal ones with
$\binom{c_i}{i}=0$, namely $c_i = i - 1$. Thus we have found the 5-combination \{8, 6, 3, 1, 0\}.

\subsection{Combinadic in GPU}

Exploit Combinadic in GPU is pretty immediate. Each thread in GPU has an unique id value that goes from zero to the number of initialized threads minus one. 
This id could be used to take a specific combination where each element from the set correspond to an id of a specific resource. 


\section{Repetition}

What explain until now is based on combination with no repetition but a resource can be instantiate more than once!

In a sequential algorithm the max repetition constraint is implemented since the beginning in the code while here it is developed in thread differently. In this case also the CPU algorithm follow the same method in order to have a better prototype of the best parallel one.

Like will be show in the next chapter, it's not possible to create all repetitions of combinations set inside a single thread on the GPU, errors will arise due to time because GPU kernel can last for a limited period of time.
To avoid this problem single thread should handle also each single repetition. 
In this way much more simpler thread are created with combinations that could not respect the area limit or that not cover all the operations. 
While in the previous version all the repetitions that respect area constraint are analysed and the useless are not created, now all ones are take into consideration.

To not explode computation a max of repetition for each resources is set.

\section{Combinadic and repetition}

To have repetition for combination given from $\binom{n}{k}$ and having a max number of repetition $max_{repetition}$, the threads that 
will be created are given by the formula \ref{tot_thred}.

\begin{equation}
    M = \binom{n}{k} * max_{repetition}^k
    \label{tot_thred}
\end{equation}

The id thread can go from id zero to $M-1$ and it is then split in two according formula \ref{id_split_1} and \ref{id_split_2}.

\begin{equation}
    id_1 = d/k 
    \label{id_split_1}
\end{equation}

\begin{equation}
    id_2 = d\mod{k}
    \label{id_split_2}
\end{equation}

From $id_1$ is extract the combination like show in section \ref{Combinadic}, while from $id_2$ the corresponding repetition.

Given $id_2$ is possible calculate the corresponding repetition of a given resources $i$ according 
the following formula \ref{repetition}.

\begin{equation}
    r_i = id_i\mod{k}
    \label{repetition}
\end{equation}

With $i$ that goes from 0 to $k-1$ and with $id_0 = id_2$. Between iteration $id_i$ change according formula \ref{change_id}.

\begin{equation}
    id_{i+1} = id_i/k
    \label{change_id}
\end{equation}